%! TEX root = ../listas_completas.tex
\section{Lista 4}
\subsection{E 2.1}
Constata-se que uma viga simples bi engastada
com seção transversal quadrada de \SI{5}{mm} x \SI{5}{mm} e
comprimento de \SI{1}{m}, que suporta uma massa de \SI{2,3}{\kilogram}
em seu ponto médio tem uma frequência natural de
vibração transversal de \SI{30}{\radian\per\second}. Determine o módulo
de Young (Elasticidade) da viga.

\subsubsection{Resolução -- 2.1}
%\todo{A resolver}
\begin{equation}\label{eq:freq_nat}
    \omega_n = \sqrt{\frac{g}{\delta_{st}}}
\end{equation}
Em que $\omega_{n}$ é frequência natural; $g$ é gravidade e $\delta_{st}$ é a
deformação.

Isolando a equação \ref{eq:freq_nat} para $\delta_{st}$ obtemos:
\[
    \delta_{st}= \frac{g}{\omega_n^2}
\]
\[
    \delta_{st}=\frac{9,81}{30} = \SI{0,0109}{m}
\]
Cálculo de $k$ (coeficiente elástico) $W=k\cdot\delta_{st}$
\[
k = \frac{W}{\delta_{st}} =\frac{2,3\cdot9,8}{0,0109} \to k =
\SI{2049,1}{N /m}
\]

O momento de inércia para uma seção transversal quadrada é:
\begin{equation}\label{eq:mom_inercia_quad}
I=\frac{a^4}{12}
\end{equation}
\[
I = \frac{0,005^4}{12} \to I = \SI{5,208d-11}{m^4}
\]
A resistência elástica equivalente para uma viga bi-engastada com carga no meio
é:
\[
    k_{eq}=\frac{192\cdot E\cdot I}{L^3} \to E= \frac{k_{eq}\cdot L^3}{192\cdot I}
\]
\[
    E=\frac{\SI{2049,1}{N /m}\cdot\si{1^3m^3}}{192\cdot\SI{5,208d-11}{m^4} }
\]
\[
E = \SI{205}{GPa}
\]
\subsection{E 2.2}%
Um corpo de massa desconhecida é colocado sobre uma mola sem peso, que se
comprime \SI{2,54}{cm}. Determine (\textbf{a}) a frequência natural de
vibração do sistema massa-mola em \si{\radian\per\second} e em \si{Hz} e
(\textbf{b}) o período natural.

\subsubsection{Resolução -- 2.2}
mola sem massa associada com um corpo de massa desconhecida.
Como a massa é desconhecida, a carga atuante será $W=9,81m\si{N}$

\textbf{item a --} Frequência natural
\[
\omega_{n}=\sqrt{\frac{g}{\delta_{st}}} =
\sqrt{\frac{9,81}{0,0254}}=\SI{19,65}{\radian\per\second}
\]
%\[
    %\omega_n=\sqrt{\frac{\si{m /s^2}}{m}}=\si{\radian\per\second}
%\]
Converter de \si{\radian\per\second} para \si{\hertz}:
\[
f_{n}=\frac{\omega_{n}}{2\pi}=\frac{19,65}{2\pi}=\SI{3,127}{\hertz}
\]
%\[
%k=\frac{W}{\delta_{st}}=\frac{mg}{\delta_{st}}
%\]
\textbf{item b --} Período natural:
\[
\tau_{n}=\frac{1}{f_{n}} = \frac{1}{3,127}=\SI{0,32}{s}
\]
\subsection{E 2.3}%

Quando um colar de \SI{3}{kg} é colocado sobre o
prato que é preso à mola de rigidez desconhecida,
observa-se que a deflexão estática adicional do prato
é de \SI{42}{mm}. Determine a constante da mola $k$ em \si{N\per\meter}.

\subsubsection{Resolução -- 2.3}

\[
\omega_{n}=\sqrt{\frac{g}{\delta_{st}}} = \sqrt{\frac{k}{m}}
\]
Elevando os dois lados da equação ao quadrado e isolando $k$, temos:
\[
\frac{g}{\delta_{st}}=\frac{k}{m} \to k=m\times \frac{g}{\delta_{st}}
\]
\[
    k = \SI{3}{kg}\times \frac{9,81\,\si{m /s^2}}{0,042\,\si{m}} \to k =
    \SI{700,7}{N /m}
\]
