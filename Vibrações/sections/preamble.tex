\usepackage[utf8]{inputenc}
\usepackage[T1]{fontenc}
\usepackage[brazil]{babel}
\usepackage{csquotes}
\usepackage[tmargin=2cm,lmargin=3cm,rmargin=2cm,bmargin=2cm]{geometry}
\usepackage{amsmath,amssymb,mathtools}
\usepackage[output-decimal-marker={,},per-mode=symbol]{siunitx}
\usepackage{cancel}
\usepackage{enumitem}
%\usepackage{icomma}
%\usepackage{titling}
\usepackage{titlesec}
\usepackage{parskip}
\usepackage{cancel}
%\usepackage{microtype}
%\usepackage{hyperref}

%\hypersetup{
%    colorlinks=true, %set true if you want colored links
%    linktoc=all,     %set to all if you want both sections and subsections linked
%    linkcolor=blue,  %choose some color if you want links to stand out
%    citecolor=black,
%    urlcolor=blue
%}

% Formata as seções de acordo com esses parâmetros em {}.
% \titleformat{command}[shape]{format}{label}{sep}{before-code}[after-code]
\titleformat{\section}{\Large\bfseries} {} {0em} {} %[\titlerule]
\titleformat{\subsection}[runin]{\bfseries\large}{}{0em}{}[ -- ]
\titleformat{\subsubsection} {\bfseries} {} {0em} {}%[\titlerule]

% Permite inserir a subseção "Resolução" com o comando \resol
\newcommand{\resol}{\subsubsection{\underline{Resolução:}}}
% alguns comandos que não devem ser usados mas estão no documento
% usar o pacote siunitx é mais confiável.
\newcommand{\kg}{\si{kg}}
\newcommand{\mm}{\si{mm}}
\newcommand{\cm}{\si{cm}}

\newlist{enumalpha}{enumerate}{1}   % Cria um novo ambiente enumerate com letras
\setlist[enumalpha]{label=\alph*)}  % pode-se criar vários sem alterar o original
